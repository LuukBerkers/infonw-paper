\documentclass[a4paper, dutch]{scrartcl}
\usepackage[utf8]{inputenc}
\usepackage[dutch]{babel}
\usepackage[automark]{scrlayer-scrpage}
\clearpairofpagestyles
\usepackage[style=ieee]{biblatex}
\addbibresource{raspberry-pi.bib}
\title{Vergelijking het Raspberry Pi- en Arduinoplatform}
\subtitle{Papergroep 34}
\subject{INFONW Paper}
\author{
    Wilmar Duvekot 6523617 \and Luuk Berkers 6793592 \and Jacob-Jan Mosselman
    6675522 \and Kenneth Man 6007767 \and Herman Horneman 6897630
}
\date{28 oktober 2019}
\ihead{\rightmark}
\ohead{INFONW Paper}
\addtokomafont{pagehead}{\upshape}

\begin{document}
\maketitle

\begin{abstract}
    Dit onderzoek probeert een antwoord te geven op de vraag of een Arduino of een Raspberry Pi een
    betere mini-computer is.
    Het onderzoek geeft daarop een antwoord en ook op de vraag wat de verschillen precies zijn.
    De Arduino is een mini-computer die vooral handig kan zijn bij het automatiseren van processen,
    en de Raspberry Pi is een iets uitgebreidere computer met wat meer aansluitingen beschikbaar.
    Het grootste verschil is dat de Raspberry Pi een grotere processor heeft dan de Arduino.
    Hieruit wordt geconcludeerd dat de Rapsberry Pi het beste gebruikt kan worden voor grote
    opdrachten en dat de Arduino geschikt is voor kleinere, simpele opdrachten.
\end{abstract}

\tableofcontents

\section{Inleiding}
In de laatste decennia heeft technologie een enorme ontwikkeling doorgemaakt.
Zo zijn de telefoons uitgevonden en zijn deze steeds kleiner, sneller en beter geworden.
Televisie heeft kleur gekregen en er zijn steeds meer zenders te zien.
Ook heeft de computer een enorme ontwikkeling doorgemaakt.
De computers van vroeger konden klaslokalen vullen, maar tegenwoordig kunnen ze in een rugzak
meegenomen worden.
De laptop wordt steeds kleiner, steeds sneller en gaat ook steeds beter presteren.
De technologie gaat tegenwoordig zelfs zo ver, dat een fotolijstje digitaal gemaakt kan worden,
zodat mensen foto's naar het lijstje kunnen sturen en dat het lijstje die foto's dan laat zien, een
voorbeeld daarvan is de Claudia digitale fotolijst \cite{innovu2019fotolijst}.

De technologie wordt steeds kleiner en beter dus, dat is ook te zien aan de mini-computers.
De Raspberry Pi \cite{raspberry2019raspberry} en de Arduino \cite{arduino2019arduino} zijn ongeveer
even groot als een telefoon en ze kunnen veel.
Het principe is hetzelfde als een laptop/computer.
De computers kunnen worden geprogrammeerd, en ze kunnen in principe alles doen, zolang het
geprogrammeerd wordt.
De Arduino en Raspberry Pi zullen in dit onderzoek verder uitgelegd worden.

Dit onderzoek zal gaan over de verschillen tussen de Arduino en Raspberry Pi, zodat duidelijk wordt
wat met de mini-computers gedaan kan worden en wat het nut kan zijn in de samenleving.
De onderzoeksvraag luidt: Wat zijn de verschillen tussen Raspberry Pi en Arduino en welke is beter?
Op deze vraag zal een antwoord gevonden worden in dit paper.
Dit paper zal beginnen met informatie over de Arduino, de architectuur, hardware, software en
toepassingen.
Vervolgens zal er worden verteld over de Raspberry Pi, architectuur, hardware, software en
toepassingen.
Daarna zullen verschillen en overeenkomsten besproken worden.
Vervolgens volgt de conclusie en de appendix.

% \section{Architectuur \& Hardware}
% Beide platformen zijn bedoeld voor gebruik in projecten waarin een computer nodig is, maar niet heel
% veel computer, dit is ook zichtbaar in het ontwerp van de hardware.
% De afmeting van de Raspberry Pi 4 \cite{raspberry2019brief} zijn vergelijkbaar met verschillende
% Arduinomodellen zoals de Arduino MEGA 2560 \cite{arduino2019mega} en de Arduino Uno
% \cite{arduino2019uno}.
% Daarnaast zijn er binnen beide platformen kleinere versies verkrijgbaar zoals de Raspberry Pi Zero
% \cite{raspberry2019buy} en de Arduino Nano \cite{arduino2019products}.
% De I/O van de Raspberry is bijvoorbeeld vergelijkbaar met de Arduino YÚN, deze heeft als een van de
% weinige Arduino's maar net als de Raspberry Pi USB en Ethernet.
% Deze Arduino is dan ook ontworpen om gebruikt te worden wanneer een internetverbinding nodig is
% \cite{arduino2019yun}.

% Arduino's zijn voorzien van een ATmega ARM microcontroller
% \cite{arduino2019uno,arduino2019leonardo,arduino2019mega,arduino2019yun,kumar2015arduino} en
% de Raspberry Pi 4 is voorzien van een quad-core Cortex-A72 ARM v8 processor
% \cite{raspberry2019brief}.
% Beide platformen gebruiken dus een ARM processor.
% ARM (Advanced RISC machine) processors zijn RISC processors in dit valt dus goed samen met de
% filosofie van deze producten om een simpele computer te zijn.

% \section{Software}
% Zoals eerder gezegd zijn beide platformen nuttig voor gebruik in projecten, zowel voor hobbyisten
% als professionele toepassingen.
% Beide platformen zijn dan ook heel makkelijk te programmeren en nuttig als middel om te leren
% programmeren \cite{raspberry2015what,jamieson2011arduino,rubio2013using}.

\printbibliography
\end{document}
